\documentclass[landscape]{report}
\title{CMSI 488 Homework 1}
\author{Don Rowe}
\date{Due February 21, 2012}

\usepackage{listings}
\usepackage[landscape]{geometry}

\lstset{
  basicstyle=\footnotesize
}

\begin{document}

  \maketitle
  
  \begin{enumerate}
    \item %1
    \begin{enumerate}
      \item %1a
        \lstinputlisting{prob1a.regex}
        Canadian postal are in the format A0A 0A0, where ``A" is any English
        capital except for D, F, I, O, Q, or U and ``0" is any digit.
        Additionally, the first letter cannot be W or Z. Also, notice the
        space between the groups of three.
      \item %1b
        \lstinputlisting{prob1b.regex}
        Visa card numbers begin with a 4 and are followed by either 12 or 15
        more digits. The above regular expression assumes that there are always
        at least the first 13 digits with an optional additional 3 digits.
      \item %1c
        \lstinputlisting{prob1c.regex}
        MasterCard numbers begin with a 5, and the second digit is between 1
        and 5. The remaining 14 digits can be any digit.
      \item %1d
        \lstinputlisting{prob1d.regex}
        Numeric literals in Ada95 come in two flavors: based literals and
        decimal literals.
        
        A based literal is separated into three sections
        delimited by \# signs. The first section is any number of digits,
        possibly separated at arbitrary intervals after the first digit by an
        underscore character; this sequence, represented by
        \lstinline`\d+(_\d+)*` in the regular expression, is simply called a
        numeral. The second section is a based numeral possibly having a
        fractional part after a dot; based numerals are like numerals, only
        they are hexadecimal and are represented by
        \lstinline`[\dA-F](_[\dA-F]+)*` in the regular expression. The third
        section is an optional exponent notated with a capital E, an optional
        plus or minus sign, and a numeral; an exponent is represented in the
        regular expression by \lstinline`(E[+-]?\d+(_\d)*)`.
        
        A decimal literal is comprised of a numeral followed by an optional
        fractional part and an optional exponent.
    \end{enumerate}
  \end{enumerate}

\end{document}